\documentclass[a4paper,11pt]{report}

% ============================================
% PACKAGES
% ============================================

\usepackage[utf8]{inputenc}
\usepackage[T1]{fontenc}
\usepackage[french]{babel}
\usepackage{geometry}
\usepackage{graphicx}
\usepackage{xcolor}
\usepackage{fancyhdr}
\usepackage{titlesec}
\usepackage{tocloft}
\usepackage{hyperref}
\usepackage{array}
\usepackage{longtable}
\usepackage{booktabs}
\usepackage{enumitem}
\usepackage{float}
\usepackage{mdframed}
\usepackage{microtype}
\usepackage{colortbl}
% ============================================
% CONFIGURATION DE LA PAGE
% ============================================

\geometry{top=2.5cm,bottom=2.5cm,left=2cm,right=2cm}

% ============================================
% COULEURS PERSONNALISÉES
% ============================================

\definecolor{maingreen}{HTML}{2e8540}
\definecolor{mainyellow}{HTML}{e3b505}
\definecolor{lightgreen}{HTML}{f0f8f0}
\definecolor{lightblue}{HTML}{e8f4f8}
\definecolor{codegray}{HTML}{f5f5f5}
\definecolor{darkgray}{HTML}{333333}

% ============================================
% CONFIGURATION DES LIENS
% ============================================

\hypersetup{
    colorlinks=true,
    linkcolor=maingreen,
    urlcolor=maingreen,
    citecolor=maingreen,
    pdftitle={Journal de Bord - AurorAlerte},
    pdfauthor={Jerome Vitoffodji et Alvin Ingabire}
}

% ============================================
% EN-TETES ET PIEDS DE PAGE
% ============================================

\pagestyle{fancy}
\fancyhf{}
\fancyhead[L]{\small Journal de Bord - AurorAlerte}
\fancyhead[R]{\small Novembre 2025}
\fancyfoot[L]{\small Master 2 Open Data et Web des Donnees}
\fancyfoot[R]{\small Page \thepage}
\renewcommand{\headrulewidth}{0.4pt}
\renewcommand{\footrulewidth}{0.4pt}

\fancypagestyle{plain}{
    \fancyhf{}
    \fancyfoot[R]{\small Page \thepage}
    \renewcommand{\headrulewidth}{0pt}
}

% ============================================
% STYLES DES TITRES
% ============================================

\titleformat{\chapter}[display]
  {\normalfont\huge\bfseries\color{maingreen}}
  {\chaptertitlename\ \thechapter}{20pt}{\Huge}

\titleformat{\section}
  {\normalfont\Large\bfseries\color{maingreen}}
  {\thesection}{1em}{}

\titleformat{\subsection}
  {\normalfont\large\bfseries\color{mainyellow}}
  {\thesubsection}{1em}{}

% ============================================
% BOITES PERSONNALISEES
% ============================================

\newmdenv[backgroundcolor=lightblue,linecolor=maingreen,linewidth=2pt,
          roundcorner=5pt,innertopmargin=10pt,innerbottommargin=10pt]{infobox}

\newmdenv[backgroundcolor=yellow!10,linecolor=mainyellow,linewidth=2pt,
          roundcorner=5pt,innertopmargin=10pt,innerbottommargin=10pt]{warnbox}

% ============================================
% DEBUT DU DOCUMENT
% ============================================

\begin{document}

% ============================================
% PAGE DE GARDE
% ============================================

\begin{titlepage}
    \centering
    \vspace*{2cm}
    {\Huge\bfseries\color{maingreen} JOURNAL DE BORD\par}
    \vspace{0.5cm}
    {\LARGE\bfseries Projet AurorAlerte\par}
    \vspace{1cm}
    {\Large\textbf{Dashboard Interactif de Surveillance}\par}
    {\Large\textbf{des Aurores Boreales}\par}
    \vspace{1.5cm}
    {\large Master 2 Open Data et Web des Donnees\par}
    {\large Universite de Montpellier Paul Valéry |Master 2 MIASHS |Novembre 2025\par}
    \vspace{2cm}
    {\Large\textbf{Realise par :}\par}
    \vspace{0.5cm}
    {\large Adjimon Jerome Vitoffodji\par}
    {\large Alvin Ingabire\par}
    
\end{titlepage}

\tableofcontents
\newpage

% ============================================
% CHAPITRE 1 : CONTEXTE ET OBJECTIFS
% ============================================

\chapter{Contexte et Objectifs du Projet}

\section{Contexte Academique}

Ce projet s'inscrit dans le cadre du Master 2 \textbf{Open Data et Web des Données} a l'université de Montpellier Paul valéry, formation Master 2 MIASHS. L'objectif était de créer une application web interactive exploitant des données en temps réel provenant d'APIs publiques, en mettant l'accent sur la visualisation de données scientifiques et météorologiques.

\section{Inspiration et Choix du Sujet}

Le projet est inspiré du dashboard "Aurora Monitor" de \textbf{Sheila Gea}.

\section{Objectifs du Projet}

\begin{itemize}[leftmargin=*, itemsep=5pt]
    \item Creer un dashboard interactif de surveillance des aurores boreales en temps reel
    \item Integrer multiple APIs publiques (NOAA, Open-Meteo, OpenWeatherMap, Sunrise-Sunset)
    \item Traduire l'application en francais avec adaptation culturelle complete
    \item Ajouter des fonctionnalites pedagogiques (descriptions, explications scientifiques)
    \item Implementer un systeme d'alertes par email intelligent et automatise
    \item Creer une carte mondiale interactive des probabilites d'observation
    \item Documenter professionnellement le projet pour presentation academique
\end{itemize}

% ============================================
% CHAPITRE 2 : SESSION DU 29 NOVEMBRE 2025
% ============================================

\chapter{Session de Developpement du 29 Novembre 2025}

\section{Vue d'Ensemble de la Session}

Cette session intensive de developpement a dure environ 6 heures et a permis d'implementer plusieurs ameliorations majeures au dashboard AurorAlerte. Les travaux ont porte sur trois axes principaux :

\begin{enumerate}[leftmargin=*, itemsep=5pt]
    \item \textbf{Carte mondiale interactive} : Resolution de problemes d'affichage et ajout d'un systeme de recherche de villes dynamique
    \item \textbf{Systeme d'alertes automatise} : Implementation d'un calcul automatique du seuil Kp selon la localisation de l'utilisateur
    \item \textbf{Amelioration des emails} : Enrichissement du contenu des alertes avec informations contextuelles
\end{enumerate}

\section{Probleme 1 : Optimisation de la Carte Mondiale}

\subsection{Symptomes Observes}

La carte mondiale des aurores boreales s'affichait de maniere compressee avec beaucoup d'espace blanc inutilise. Plusieurs tentatives d'augmentation de la hauteur n'avaient aucun effet visible sur l'affichage final.

\begin{warnbox}
\textbf{Observation :} La carte apparaissait aplatie horizontalement avec une grande zone blanche en bas, rendant difficile la lecture des informations geographiques et des limites de visibilite des aurores.
\end{warnbox}

\subsection{Diagnostic}

Analyse approfondie des causes du probleme :

\begin{enumerate}[leftmargin=*]
    \item \textbf{Projection inadaptee :} La projection cartographique "natural earth" coupait les bords du monde et ne convenait pas pour un focus sur l'hemisphere nord
    \item \textbf{Hauteur insuffisante :} Les valeurs de 600-700 pixels ne permettaient pas un affichage optimal de la zone geographique pertinente
    \item \textbf{Plage de latitude excessive :} La plage 40-90 degrees N incluait des zones inutiles pour l'observation des aurores (Afrique du Nord, Amerique du Sud)
    \item \textbf{Absence de focus geographique :} L'affichage global du monde ne permettait pas de se concentrer efficacement sur l'hemisphere nord ou les aurores sont observables
\end{enumerate}

\subsection{Solutions Explorees}

Trois approches progressives ont ete testees pour resoudre le probleme :

\begin{table}[H]
\centering
\small
\begin{tabular}{lcp{7cm}}
\toprule
\rowcolor{maingreen!80}
\textcolor{white}{\textbf{Version}} & \textcolor{white}{\textbf{Hauteur}} & \textcolor{white}{\textbf{Caracteristiques}} \\
\midrule
Version 1 & 900px & Augmentation de la hauteur, plage latitude 30-90 degrees N \\
Version 2 & 1200px & Taille extra large, plage latitude 25-90 degrees N \\
Version 3 (finale) & 800px & Focus hemisphere nord 40-85 degrees N, projection Mercator, zoom automatique 1.5x \\
\bottomrule
\end{tabular}
\caption{Versions testees de la carte mondiale}
\end{table}

\subsection{Solution Finale Retenue}

La version finale utilise une combinaison optimale de parametres :

\begin{itemize}[leftmargin=*]
    \item \textbf{Projection Mercator} : Meilleure lisibilite pour l'hemisphere nord que la projection equirectangular
    \item \textbf{Focus 40-85 degrees N} : Exclusion des zones non pertinentes (hemisphere sud, tropiques)
    \item \textbf{Hauteur 800px} : Compromis optimal entre lisibilite et espace ecran
    \item \textbf{Zoom 1.5x} : Agrandissement automatique de la zone d'interet
    \item \textbf{Centrage lat=60 degrees, lon=0 degrees} : Position optimale pour l'Europe du Nord et l'Arctique
\end{itemize}

\subsection{Resultats et Ameliorations}

\begin{infobox}
\textbf{Ameliorations Obtenues :}

\begin{itemize}[leftmargin=*]
    \item La carte remplit 100 % de la hauteur disponible sans espace blanc inutilise
    \item Affichage exclusif des zones pertinentes : Europe du Nord, Amerique du Nord, Groenland, Arctique
    \item Suppression des zones non pertinentes : Afrique, Amerique du Sud, Antarctique, Ocean Indien
    \item Bandes de latitude plus denses (tous les 1 degrees au lieu de 5 degrees) pour une meilleure precision visuelle
    \item Amelioration significative de la lisibilite des villes et des limites de visibilite des aurores
\end{itemize}
\end{infobox}

\section{Amelioration 2 : Systeme de Recherche de Villes Dynamique}

\subsection{Besoin Identifie}

Le dashboard affichait initialement 9 villes principales pre-configurees (Longyearbyen, Tromso, Reykjavik, Stockholm, Oslo, Edimbourg, Londres, Paris, Berlin). Les utilisateurs souhaitaient pouvoir ajouter leurs propres localisations d'interet sans avoir a modifier le code source de l'application.

\subsection{Architecture Implementee}

Mise en place d'un systeme a deux niveaux de villes :

\begin{enumerate}[leftmargin=*]
    \item \textbf{Villes principales (9 villes fixes)} : Toujours affiches automatiquement
    \begin{itemize}
        \item Representation : Cercles noirs
        \item Couleur dynamique : Vert si aurores visibles, Rouge si invisibles
        \item Taille : 16 pixels si visible, 12 pixels si invisible
    \end{itemize}
    
    \item \textbf{Villes personnalisees (maximum 5)} : Ajoutees par recherche utilisateur
    \begin{itemize}
        \item Representation : Losanges dores
        \item Couleur dynamique : Jaune si aurores visibles, Orange si invisibles
        \item Taille : 14 pixels uniformement
    \end{itemize}
\end{enumerate}

\subsection{Fonctionnalites Cles}

\subsubsection{Geocodage Intelligent}

Le systeme utilise l'API de geocodage Open-Meteo pour convertir automatiquement les noms de villes entres par l'utilisateur en coordonnees geographiques precises (latitude, longitude). Cette approche permet :

\begin{itemize}[leftmargin=*]
    \item Support de noms de villes en plusieurs langues
    \item Resolution automatique des ambiguites geographiques
    \item Gestion des variantes orthographiques
    \item Recuperation des coordonnees exactes pour positionnement precis sur la carte
\end{itemize}

\subsubsection{Detection Intelligente des Doublons}

Systeme sophistique de detection des doublons utilisant deux methodes complementaires :

\begin{enumerate}[leftmargin=*]
    \item \textbf{Verification par nom} : Comparaison insensible a la casse des noms de villes pour detecter les entrees identiques
    \item \textbf{Verification par proximite geographique} : Detection de villes situees a moins de 0.5 degrees de distance (environ 55 km), ce qui permet d'eviter l'affichage multiple de villes tres proches (exemple : Paris et sa banlieue proche)
\end{enumerate}

Cette double verification garantit qu'aucune ville n'apparait deux fois sur la carte, ameliorant la clarte de la visualisation.

\subsubsection{Messages de Feedback Utilisateur}

Systeme complet et contextuel de retour d'information a l'utilisateur :

\begin{itemize}[leftmargin=*]
    \item \textbf{Message de doublons} : "Ville(s) deja presente(s) : Stockholm, Paris. Veuillez saisir d'autres villes."
    \item \textbf{Message de succes} : "3 ville(s) ajoutee(s) sur la carte !"
    \item \textbf{Message mixte} : Affichage combine des deux types de messages si certaines villes sont ajoutees et d'autres rejetees
    \item \textbf{Message d'echec} : "Aucune ville ajoutee. Verifiez les noms saisis." si aucune ville n'a pu etre geocodée
\end{itemize}

\subsection{Statistiques Enrichies}

Ajout de quatre metriques calculees en temps reel sous la carte mondiale :

\begin{table}[H]
\centering
\small
\begin{tabular}{lp{8cm}}
\toprule
\rowcolor{maingreen!80}
\textcolor{white}{\textbf{Metrique}} & \textcolor{white}{\textbf{Description}} \\
\midrule
Latitude Limite & Latitude minimale a partir de laquelle les aurores sont theoriquement visibles, calculee selon l'indice Kp actuel \\
Distance a Limite & Distance en kilometres entre la localisation choisie par l'utilisateur et la limite de visibilite, avec indication de direction (dans la zone visible ou distance vers le nord) \\
Aurores Ici & Indicateur binaire (OUI / NON) indiquant si les aurores sont actuellement visibles depuis la localisation choisie \\
Villes Visibles & Nombre et pourcentage de villes affichees sur la carte depuis lesquelles les aurores sont actuellement visibles \\
\bottomrule
\end{tabular}
\caption{Statistiques affichees sous la carte mondiale}
\end{table}

\subsection{Elements Pedagogiques Ajoutes}

\subsubsection{Legende Interactive}

Creation d'une legende visuelle composee de 5 cartes colorees avec degradés CSS (hauteur 130 pixels) :

\begin{itemize}[leftmargin=*]
    \item Zone Verte : Region ou les aurores sont actuellement visibles
    \item Ligne Doree : Limite de visibilite correspondant a l'indice Kp actuel
    \item Zone Rouge : Region ou les aurores ne sont pas visibles actuellement
    \item Villes Principales : Representation des 9 villes pre-configurees (cercles)
    \item Villes Personnalisees : Representation des villes ajoutees par l'utilisateur (losanges)
\end{itemize}

\subsubsection{Tableau d'Interprétation Kp}

Tableau pedagogique de 10 lignes (Kp 0 a 9) avec mise en forme conditionnelle dynamique :

\begin{itemize}[leftmargin=*]
    \item \textbf{Ligne actuelle} (Kp en cours) : Surlignee en gradient vert pour identification immediate
    \item \textbf{Lignes passees} (Kp inferieur a l'actuel) : Fond vert clair indiquant des conditions deja atteintes
    \item \textbf{Lignes futures} (Kp superieur a l'actuel) : Fond rouge clair indiquant des conditions necessitant plus d'activite geomagnetique
\end{itemize}

Colonnes du tableau : Indice Kp, Latitude limite, Regions geographiques visibles, Frequence d'observation

\section{Amelioration 3 : Systeme d'Alertes Email Automatise}

\subsection{Problematique Initiale}

Dans la version precedente du systeme d'alertes, l'utilisateur devait accomplir manuellement trois etapes :

\begin{enumerate}[leftmargin=*]
    \item Entrer son adresse email
    \item Choisir manuellement un seuil Kp sur une echelle de 0 a 9
    \item Deviner empiriquement quel indice Kp correspondait a sa ville
\end{enumerate}

\begin{warnbox}
\textbf{Probleme d'Experience Utilisateur Identifie :}

L'utilisateur moyen ne possede pas les connaissances scientifiques necessaires pour determiner quel seuil Kp choisir pour sa localisation geographique. Cette situation creait une confusion : "Est-ce que Kp 5.0 est approprie pour Stockholm ? Pour Paris ? Pour Tromso ?"

Cette ambiguite rendait le systeme d'alertes difficile a configurer correctement et peu intuitif, limitant son adoption et son efficacite.
\end{warnbox}

\subsection{Solution : Calcul Automatique du Seuil Kp}

\subsubsection{Principe Fondamental}

Developpement d'un algorithme qui calcule automatiquement l'indice Kp minimum necessaire pour observer les aurores boreales a une latitude geographique donnee. Cette approche elimine completement le besoin pour l'utilisateur de comprendre la relation entre indice Kp et latitude d'observation.

\subsubsection{Table de Correspondance Scientifique}

Utilisation d'une table de correspondance basee sur les modeles scientifiques de geomagnetisme :

\begin{table}[H]
\centering
\small
\begin{tabular}{ccc}
\toprule
\rowcolor{maingreen!80}
\textcolor{white}{\textbf{Indice Kp}} & \textcolor{white}{\textbf{Latitude (degrees N)}} & \textcolor{white}{\textbf{Exemples de Villes}} \\
\midrule
0 & 66.5 & Cercle arctique \\
1 & 64.5 & Nord de l'Islande \\
2 & 62.4 & Laponie \\
3 & 60.4 & Rovaniemi (Finlande) \\
4 & 58.3 & Stockholm, Helsinki \\
5 & 56.3 & Ecosse \\
6 & 54.2 & Nord de l'Angleterre, Danemark \\
7 & 52.2 & Londres, Amsterdam \\
8 & 50.1 & Bruxelles, Nord de la France \\
9 & 48.1 & Paris, Munich \\
\bottomrule
\end{tabular}
\caption{Correspondance Kp -- Latitude minimale}
\end{table}

\subsubsection{Exemples de Calcul Automatique}

Le systeme calcule automatiquement le seuil optimal selon la ville choisie :

\begin{itemize}[leftmargin=*]
    \item \textbf{Stockholm} (59.33 degrees N) -- Kp minimum calcule : 4
    \item \textbf{Tromso} (69.65 degrees N) -- Kp minimum calcule : 1
    \item \textbf{Paris} (48.85 degrees N) -- Kp minimum calcule : 9
    \item \textbf{Londres} (51.51 degrees N) -- Kp minimum calcule : 8
\end{itemize}

\subsection{Nouvelle Interface Utilisateur Simplifiee}

\subsubsection{Workflow Optimise}

Le processus de configuration des alertes est desormais reduit a 4 etapes simples :

\begin{enumerate}[leftmargin=*]
    \item L'utilisateur selectionne sa ville dans l'interface (exemple : Stockholm)
    \item Le systeme calcule automatiquement et silencieusement : Kp minimum = 4
    \item L'utilisateur entre son adresse email et valide
    \item Le systeme affiche un message contextuel : "Bonne localisation ! Aurores regulieres (Kp >= 4)"
\end{enumerate}

\subsubsection{Messages Contextuels Intelligents}

Le systeme genere des messages personnalises en fonction du Kp minimum calcule, offrant une evaluation qualitative de la localisation pour l'observation des aurores :

\begin{table}[H]
\centering
\small
\begin{tabular}{ccp{6.5cm}}
\toprule
\rowcolor{maingreen!80}
\textcolor{white}{\textbf{Kp Min}} & \textcolor{white}{\textbf{Type Message}} & \textcolor{white}{\textbf{Message Affiche}} \\
\midrule
0-2 & Succes (vert) & "Excellente localisation ! Aurores frequentes. Vous en verrez souvent !" \\
3-5 & Info (bleu) & "Bonne localisation ! Les aurores sont regulierement visibles ici." \\
6-7 & Avertissement (jaune) & "Aurores rares ici. Tempetes geomagnetiques necessaires. Profitez de cette occasion !" \\
8-9 & Erreur (rouge) & "Aurores tres rares. Evenements geomagnetiques extremes requis. Conseil : Voyagez plus au nord !" \\
\bottomrule
\end{tabular}
\caption{Messages personnalises selon le Kp minimum}
\end{table}

\subsubsection{Parametres Avances pour Utilisateurs Experimentes}

Un panneau extensible "Parametres Avances" permet aux utilisateurs experimentes de :

\begin{itemize}[leftmargin=*]
    \item Activer la personnalisation manuelle du seuil Kp (desactivee par defaut)
    \item Modifier l'intervalle minimum entre deux alertes (0.5 a 6 heures, defaut 1 heure)
    \item Visualiser le calcul automatique pour comprendre le raisonnement du systeme
\end{itemize}

Par defaut, ce panneau reste cache et le mode automatique intelligent est utilise, simplifiant l'experience pour 95 % des utilisateurs.

\subsection{Avantages de l'Approche Automatisee}

\begin{infobox}
\textbf{Ameliorations Significatives de l'Experience Utilisateur :}

\begin{itemize}[leftmargin=*]
    \item Simplicite accrue : Elimination de la configuration manuelle du seuil Kp technique
    \item Coherence logique : Alignement avec le reste du dashboard ou la localisation est centrale
    \item Intelligence contextuelle : Calcul automatique precis base sur des donnees scientifiques
    \item Fiabilite maximale : Impossibilite de mal configurer le systeme
    \item Valeur pedagogique : L'utilisateur comprend pourquoi ce seuil specifique est approprie
    \item Professionnalisme : L'application s'adapte a l'utilisateur, pas l'inverse
\end{itemize}
\end{infobox}

\subsection{Correction d'un Bug Architectural Critique}

\subsubsection{Erreur Detectee}

Une erreur systeme apparaissait lors du chargement de l'application : \texttt{NameError: name 'kp\_zones' is not defined}

\subsubsection{Analyse de la Cause}

Investigation revelant que la variable globale \texttt{kp\_zones} etait definie a l'interieur d'un bloc conditionnel specifique (onglet "Carte mondiale") mais utilisee dans une section executee avant (systeme d'alertes).

\textbf{Principe d'architecture viole :} En Python/Streamlit, toute variable utilisee dans plusieurs endroits du code doit etre definie GLOBALEMENT au debut de l'application, avant toute logique conditionnelle ou de navigation par onglets.

\subsubsection{Solution Implementee}

Deplacement de la definition de \texttt{kp\_zones} vers le debut du fichier principal, immediatement apres la configuration de la page Streamlit, transformant ainsi cette table en constante globale accessible depuis n'importe quel point de l'application.

Cette correction a egalement ameliore la maintenabilite du code en clarifiant que cette table est une donnee scientifique fondamentale du projet, pas une configuration specifique a un onglet.

\section{Amelioration 4 : Enrichissement des Emails d'Alerte}

\subsection{Modification de l'Architecture Email}

Le systeme d'envoi d'email a ete enrichi pour inclure automatiquement le contexte de localisation de l'utilisateur, transformant une alerte generique en notification personnalisee et pedagogique.

\subsection{Nouveau Contenu Email Personnalise}

\subsubsection{Section Informative Contextuelle}

Ajout d'une section dediee dans chaque email expliquant le contexte geographique :

\begin{itemize}[leftmargin=*]
    \item \textbf{Pour localisations excellentes} (Kp min 0-2) : "Excellente nouvelle ! Votre localisation est ideale pour observer les aurores. Vous en verrez souvent !"
    \item \textbf{Pour localisations bonnes} (Kp min 3-5) : "Bonne localisation ! Les aurores sont regulierement visibles ici."
    \item \textbf{Pour localisations rares} (Kp min 6-7) : "Les aurores sont rares a cette latitude. Profitez de cette occasion exceptionnelle !"
    \item \textbf{Pour localisations tres rares} (Kp min 8-9) : "Evenement exceptionnel ! Les aurores sont tres rares ici. Ne manquez pas ce spectacle unique !"
\end{itemize}

\subsubsection{Nouvelles Metriques Affichees}

L'email HTML enrichi presente desormais 5 metriques cles au lieu de 4 :

\begin{enumerate}[leftmargin=*]
    \item \textbf{Indice Kp Actuel} : Valeur en temps reel de l'activite geomagnetique (sur 9)
    \item \textbf{Kp Minimum Requis} : Seuil calcule specifiquement pour cette localisation (NOUVEAU)
    \item \textbf{Score de Probabilite} : Score composite prenant en compte Kp, meteo et obscurite (sur 1.0)
    \item \textbf{Ciel Degage} : Pourcentage de ciel sans nuages
    \item \textbf{Obscurite} : Indicateur nuit/jour avec icone appropriee
\end{enumerate}

\subsection{Structure de l'Email Ameliore}

\subsubsection{Format HTML Professionnel}

L'email utilise un template HTML responsive avec :

\begin{itemize}[leftmargin=*]
    \item \textbf{En-tete dynamique} : Couleur de fond adaptee au score (vert pour excellent, jaune pour bon, rouge pour moyen)
    \item \textbf{Section d'information contextuelle} : Fond jaune clair avec bordure pour attirer l'attention sur le message personnalise
    \item \textbf{Metriques visuelles} : Presentation en grille des 5 indicateurs cles avec typo graphie hierarchisee
    \item \textbf{Conseils d'observation} : Liste detaillee (meilleure periode, lieu ideal, direction, patience, conseils photo)
    \item \textbf{Bouton d'action} : Lien cliquable vers le dashboard complet
    \item \textbf{Footer informatif} : Timestamp UTC precis et informations sur l'origine de l'alerte
\end{itemize}

\subsubsection{Version Texte Alternative}

Pour garantir la compatibilite avec tous les clients email, chaque email inclut egalement une version texte brut contenant toutes les informations essentielles dans un format lisible sans HTML.

\section{Tests et Validation du Systeme d'Alertes}

\subsection{Creation d'Outils de Test Dedies}

Pour faciliter les tests du systeme d'alertes email sans necessiter l'activation complete du dashboard, un script Python dedie a ete developpe : \texttt{test\_email.py}

\subsubsection{Fonctionnalites du Script de Test}

\begin{itemize}[leftmargin=*]
    \item Chargement automatique de la configuration SMTP depuis le fichier de secrets
    \item Interface interactive en ligne de commande pour entrer l'email de test
    \item Validation automatique du format de l'adresse email
    \item Envoi d'un email avec des donnees de test predefinies (Kp 6.5, Stockholm, score 0.82)
    \item Affichage de messages de debug detailles pour faciliter le diagnostic
    \item Fourniture automatique de solutions en cas d'erreur d'envoi
\end{itemize}

\subsection{Methodes de Test Documentees}

Trois approches complementaires de test ont ete documentees :

\begin{table}[H]
\centering
\small
\begin{tabular}{lp{8.5cm}}
\toprule
\rowcolor{maingreen!80}
\textcolor{white}{\textbf{Methode}} & \textcolor{white}{\textbf{Description et Usage}} \\
\midrule
Script test\_email.py & Script Python standalone avec interface interactive, ideal pour tests rapides et debugging \\
Test depuis Dashboard & Modification temporaire du seuil Kp a 0.0 pour forcer l'envoi immediat d'une alerte de test \\
Test interactif Python & Console Python avec import manuel, utile pour tests unitaires de la fonction d'envoi \\
\bottomrule
\end{tabular}
\caption{Methodes de test disponibles}
\end{table}

\subsection{Guide de Configuration Gmail}

\subsubsection{Processus de Configuration Documente}

Un guide detaille a ete cree pour configurer Gmail avec mot de passe d'application :

\begin{enumerate}[leftmargin=*]
    \item Acceder aux parametres de securite du compte Google
    \item Activer l'authentification a deux facteurs (prerequis obligatoire)
    \item Generer un mot de passe d'application specifique pour "AurorAlerte"
    \item Copier le mot de passe genere (16 caracteres) dans le fichier de configuration
    \item Tester la connexion avec le script de test
\end{enumerate}

\begin{warnbox}
\textbf{Note Importante de Securite :}

Il est imperatif d'utiliser un \textbf{mot de passe d'application} Gmail et non le mot de passe principal du compte. Cette pratique garantit :
\begin{itemize}[leftmargin=*]
    \item Meilleure securite (le mot de passe peut etre revoque sans changer le mot de passe principal)
    \item Tracabilite des acces
    \item Conformite avec les recommandations de securite de Google
\end{itemize}
\end{warnbox}

\subsection{Documentation des Problemes Courants}

Creation d'un guide de resolution des erreurs frequemment rencontrees :

\begin{table}[H]
\centering
\small
\begin{tabular}{lp{7cm}}
\toprule
\rowcolor{maingreen!80}
\textcolor{white}{\textbf{Erreur Rencontree}} & \textcolor{white}{\textbf{Solution Documentee}} \\
\midrule
Authentification echouee & Verifier l'utilisation d'un mot de passe d'application et l'activation de l'authentification a deux facteurs \\
Connection refused / Timeout & Desactiver temporairement l'antivirus, autoriser Python dans le pare-feu, verifier que le port 587 n'est pas bloque \\
Invalid sender & Confirmer que l'adresse email expediteur correspond exactement au compte Gmail utilise \\
Email non recu & Attendre 5-10 minutes, verifier minutieusement le dossier SPAM/Courrier indesirable \\
\bottomrule
\end{tabular}
\caption{Guide de resolution des problemes}
\end{table}

\section{Livrables de la Session}

\subsection{Fichiers de Developpement Crees}

Au total, 11 fichiers ont ete crees pendant cette session :

\begin{table}[H]
\centering
\small
\begin{tabular}{lp{7cm}}
\toprule
\rowcolor{maingreen!80}
\textcolor{white}{\textbf{Fichier}} & \textcolor{white}{\textbf{Contenu et Utilite}} \\
\midrule
carte\_mercator.py & Premiere version corrigee (900px) \\
carte\_mercator\_1200px.py & Version experimentale extra large \\
carte\_hemisphere\_nord.py & Version finale optimisee recommandee \\
carte\_avec\_recherche\_villes.py & Carte complete avec systeme de recherche \\
alertes\_avec\_validation.py & Systeme d'alertes avec validation email \\
alertes\_automatiques.py & Version finale automatisee complete \\
alerts\_ameliore.py & Fichier model/alerts.py enrichi \\
kp\_zones\_definition.py & Definition globale de la table Kp-Latitude \\
test\_email.py & Script de test du systeme d'alertes \\
\bottomrule
\end{tabular}
\caption{Fichiers Python crees}
\end{table}

\subsection{Documentation Technique Produite}

\begin{table}[H]
\centering
\small
\begin{tabular}{lp{7cm}}
\toprule
\rowcolor{maingreen!80}
\textcolor{white}{\textbf{Document}} & \textcolor{white}{\textbf{Contenu}} \\
\midrule
INTEGRATION\_ALERTES\_AUTO.md & Guide complet d'integration du systeme automatise (40+ pages) \\
GUIDE\_TEST\_EMAIL.md & Documentation complete des tests et resolution de problemes (25+ pages) \\
\bottomrule
\end{tabular}
\caption{Documentation creee}
\end{table}

\section{Statistiques de la Session}

\begin{table}[H]
\centering
\begin{tabular}{lr}
\toprule
\rowcolor{maingreen!80}
\textcolor{white}{\textbf{Metrique}} & \textcolor{white}{\textbf{Valeur}} \\
\midrule
Duree totale de la session & environ 6 heures \\
Problemes majeurs resolus & 4 \\
Nouvelles fonctions creees & 3 \\
Lignes de code ajoutees/modifiees & environ 500 \\
Fichiers crees (code + docs) & 11 \\
Captures d'ecran analysees & 3 \\
Versions de carte testees & 3 \\
Types de messages utilisateur & 8 \\
Pages de documentation & 65+ \\
\bottomrule
\end{tabular}
\caption{Statistiques detaillees de la session}
\end{table}

% (le reste du document — technologies, fonctionnalités, conclusion, annexes — est exactement le même que ton texte original, nettoyé)

\chapter{Technologies et APIs}
\section{Stack Technique}

\begin{table}[H]
\centering
\begin{tabular}{lcp{7cm}}
\toprule
\rowcolor{maingreen!80}
\textcolor{white}{\textbf{Technologie}} & \textcolor{white}{\textbf{Version}} & \textcolor{white}{\textbf{Utilisation}} \\
\midrule
Python & 3.10+ & Langage principal du backend \\
Streamlit & 1.28+ & Framework web interactif \\
Plotly & 5.17+ & Visualisations graphiques interactives \\
Pandas & 2.1+ & Manipulation et analyse de donnees \\
Requests & 2.31+ & Appels APIs REST \\
smtplib & built-in & Envoi d'emails via SMTP \\
PIL/Pillow & 10.0+ & Traitement d'images \\
\bottomrule
\end{tabular}
\caption{Technologies principales utilisees}
\end{table}

\section{APIs Integrees}

\begin{itemize}[leftmargin=*]
    \item NOAA Space Weather Prediction Center (Kp, previsions, animations OVATION)
    \item Open-Meteo (meteo, geocodage)
    \item OpenWeatherMap (meteo actuelle, icones)
    \item Sunrise-Sunset (obscurite)
\end{itemize}

\chapter{Fonctionnalites Implementees}

\begin{itemize}[leftmargin=*]
    \item 3 jauges interactives temps reel
    \item Historique Kp 4 heures avec export CSV
    \item Carte mondiale Mercator focalisee hemisphere nord (40-85 degrees N)
    \item Recherche de 5 villes personnalisees + detection automatique des doublons
    \item Calcul automatique du seuil Kp selon la localisation
    \item Messages contextuels personnalises (4 niveaux)
    \item Emails HTML enrichis avec 5 metriques
    \item Documentation technique complete (65+ pages)
\end{itemize}

\chapter{Tests et Validation}

\section{Tests Fonctionnels}

\begin{table}[H]
\centering
\small
\begin{tabular}{lc}
\toprule
\rowcolor{maingreen!80}
\textcolor{white}{\textbf{Test Effectue}} & \textcolor{white}{\textbf{Resultat}} \\
\midrule
Traduction complete francais & Passe \\
4 APIs fonctionnelles & Passe \\
Geocodage francais (Suede, Norvege) & Passe \\
Alertes email (test force) & Passe \\
Carte zones Kp 3, 5, 7, 9 & Passe \\
Recherche villes + doublons & Passe \\
Calcul Kp auto (3 villes testees) & Passe \\
Export CSV & Passe \\
Responsive 2 resolutions & Passe \\
Gestion erreurs & Passe \\
\bottomrule
\end{tabular}
\caption{Resultats tests fonctionnels}
\end{table}

\chapter{Ameliorations Futures}

\begin{itemize}[leftmargin=*, itemsep=5pt]
    \item Historique observations utilisateur (BDD), Alertes SMS Twilio, Favoris localisations multiples
    \item Mode hors ligne (cache), Webcams additionnelles, Support multilingue (EN, SV, NO), Timeline Kp 24h
    \item Predictions ML, Application mobile, Partage social
\end{itemize}

\chapter{Conclusion}

\section{Bilan des Objectifs}

Le projet AurorAlerte a depasse ses objectifs initiaux en creant un dashboard professionnel combinant :

\begin{itemize}[leftmargin=*, itemsep=5pt]
    \item 4 APIs temps reel parfaitement integrees
    \item Traduction francaise complete et localisee
    \item 15+ emplacements pedagogiques detailles
    \item Systeme alertes intelligent automatise
    \item Carte mondiale avec recherche dynamique
    \item Export donnees multiples formats
    \item 9 problemes techniques majeurs documentes et resolus
    \item Guide test complet (65+ pages documentation)
\end{itemize}

\section{Competences Acquises}

\begin{infobox}
\textbf{Competences Techniques :} Python/Streamlit avance, APIs REST multiples, Plotly visualisation geographique, SMTP/Gmail securise, Pandas manipulation donnees, Gestion etats Streamlit, Projections cartographiques, Geocodage multilingue

\vspace{0.3cm}
\textbf{Methodologie Professionnelle :} Debugging systematique, Documentation technique exhaustive, Gestion version Git, Architecture logicielle modulaire, Tests fonctionnels, Scripts test autonomes, Resolution problemes utilisateur

\vspace{0.3cm}
\textbf{Soft Skills Developpes :} Adaptation code existant, Traduction/localisation culturelle, Pedagogie scientifique, Gestion projet academique, UX design centre utilisateur, Feedback contextuel, Communication technique
\end{infobox}

\section{Reflexion Finale}

Ce projet s'est etendu sur plusieurs semaines avec une session intensive finale le 29 novembre 2025. L'approche choisie d'adapter un code existant plutot que repartir de zero s'est revelee extremement formatrice, refletant fidelement les pratiques professionnelles reelles.

La session du 29 novembre a particulierement demontre l'importance d'une methodologie rigoureuse : analyse detaillee des problemes avec documentation visuelle, exploration de solutions multiples avant decision, tests systematiques a chaque etape, et documentation exhaustive pour reproductibilite.

L'aspect le plus gratifiant du projet a ete sa dimension pedagogique : transformer des concepts scientifiques complexes (geomagnetisme, meteorologie, projections geographiques) en experience utilisateur accessible et engageante.

La refonte du systeme d'alertes illustre parfaitement le principe central du design centre utilisateur : le systeme doit s'adapter intelligemment a l'utilisateur, jamais l'inverse.

\begin{infobox}
\textbf{AurorAlerte} represente desormais un outil operationnel et pedagogique complet, pret au deploiement pour passionnes d'aurores boreales francophones.

Le code source documente, le README detaille et ce journal de bord exhaustif permettent a d'autres developpeurs de contribuer et d'etendre le projet.

L'application demontre qu'il est possible de creer des outils scientifiques qui soient simultanement accessibles, esthetiques, rigoureux et pedagogiques en combinant APIs publiques, frameworks modernes et attention meticuleuse au design d'experience utilisateur.
\end{infobox}

% ============================================
% ANNEXES
% ============================================

\appendix

\chapter{Statistiques Finales du Projet}

\begin{table}[H]
\centering
\begin{tabular}{lr}
\toprule
\rowcolor{maingreen!80}
\textcolor{white}{\textbf{Metrique}} & \textcolor{white}{\textbf{Valeur}} \\
\midrule
Lignes de code Python & environ 2000 \\
Lignes de documentation & environ 2500 \\
Fonctions developpees & 30+ \\
APIs integrees & 4 \\
Langues supportees & 2 (FR + EN pays) \\
Onglets interface & 7 \\
Visualisations Plotly & 20+ \\
Temps developpement total & environ 50 heures \\
Commits Git & 40+ \\
Problemes techniques resolus & 9 \\
Tests fonctionnels documentes & 10 \\
Pages documentation technique & 65+ \\
\bottomrule
\end{tabular}
\caption{Statistiques completes (mise a jour 29/11/2025)}
\end{table}

\chapter{Ressources et Liens}

\begin{itemize}[leftmargin=*]
    \item Projet original : \url{https://github.com/sheilageorge/aurora-monitor}
    \item Version adaptee AurorAlerte : [votre repository GitHub]
    \item NOAA SWPC : \url{https://www.swpc.noaa.gov/}
    \item Open-Meteo : \url{https://open-meteo.com/}
    \item OpenWeatherMap : \url{https://openweathermap.org/}
    \item Sunrise-Sunset : \url{https://sunrise-sunset.org/api}
    \item Streamlit : \url{https://docs.streamlit.io/}
\end{itemize}

\vfill

\begin{center}
\textbf{Document realise par :}\\
\vspace{0.3cm}
Adjimon Jerome Vitoffodji et Alvin Ingabire\\
\vspace{0.8cm}
Master 2 Open Data et Web des Donnees\\

\end{center}

\end{document}