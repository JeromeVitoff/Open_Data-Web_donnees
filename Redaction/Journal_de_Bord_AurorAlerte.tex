\documentclass[a4paper,11pt]{report}

% ============================================
% PACKAGES
% ============================================

\usepackage[utf8]{inputenc}
\usepackage[T1]{fontenc}
\usepackage[french]{babel}
\usepackage{geometry}
\usepackage{graphicx}
\usepackage{xcolor}
\usepackage{fancyhdr}
\usepackage{titlesec}
\usepackage{tocloft}
\usepackage{hyperref}
\usepackage{array}
\usepackage{longtable}
\usepackage{booktabs}
\usepackage{enumitem}
\usepackage{float}
\usepackage{mdframed}
\usepackage{microtype}
\usepackage{colortbl}

% ============================================
% CONFIGURATION DE LA PAGE
% ============================================

\geometry{top=2.5cm,bottom=2.5cm,left=2cm,right=2cm}

% ============================================
% COULEURS PERSONNALISÉES
% ============================================

\definecolor{mainblue}{HTML}{2C5AA0}
\definecolor{accentblue}{HTML}{4A90E2}
\definecolor{lightgray}{HTML}{f5f5f5}
\definecolor{verylightblue}{HTML}{E8F4F8}
\definecolor{codegray}{HTML}{f5f5f5}
\definecolor{darkgray}{HTML}{333333}

% ============================================
% CONFIGURATION DES LIENS
% ============================================

\hypersetup{
    colorlinks=true,
    linkcolor=mainblue,
    urlcolor=mainblue,
    citecolor=mainblue,
    pdftitle={Journal de Bord - AurorAlerte},
    pdfauthor={Jérôme Vitoffodji et Alvin Ingabire}
}

% ============================================
% EN-TÊTES ET PIEDS DE PAGE
% ============================================

\pagestyle{fancy}
\fancyhf{}
\fancyhead[L]{\small Journal de Bord - AurorAlerte}
\fancyhead[R]{\small Novembre 2025}
\fancyfoot[L]{\small Master 2 MIASHS, module Open Data et Web des Données,  \url{https://web-production-ff2d6.up.railway.app/#aura-hunter}}
\fancyfoot[R]{\small Page \thepage}
\renewcommand{\headrulewidth}{0.4pt}
\renewcommand{\footrulewidth}{0.4pt}

\fancypagestyle{plain}{
    \fancyhf{}
    \fancyfoot[R]{\small Page \thepage}
    \renewcommand{\headrulewidth}{0pt}
}

% ============================================
% STYLES DES TITRES
% ============================================

\titleformat{\chapter}[display]
  {\normalfont\huge\bfseries\color{mainblue}}
  {\chaptertitlename\ \thechapter}{20pt}{\Huge}

\titleformat{\section}
  {\normalfont\Large\bfseries\color{mainblue}}
  {\thesection}{1em}{}

\titleformat{\subsection}
  {\normalfont\large\bfseries\color{accentblue}}
  {\thesubsection}{1em}{}

% ============================================
% BOÎTES PERSONNALISÉES
% ============================================

\newmdenv[backgroundcolor=verylightblue,linecolor=mainblue,linewidth=2pt,
          roundcorner=5pt,innertopmargin=10pt,innerbottommargin=10pt]{infobox}

\newmdenv[backgroundcolor=lightgray,linecolor=accentblue,linewidth=2pt,
          roundcorner=5pt,innertopmargin=10pt,innerbottommargin=10pt]{warnbox}

% ============================================
% DÉBUT DU DOCUMENT
% ============================================

\begin{document}

% ============================================
% PAGE DE GARDE
% ============================================

\begin{titlepage}
    \centering
    \vspace*{2cm}
    {\Huge\bfseries\color{mainblue} JOURNAL DE BORD\par}
    \vspace{0.5cm}
    {\LARGE\bfseries Projet AurorAlerte\par}
    \vspace{1cm}
    {\Large\textbf{Dashboard Interactif de Surveillance}\par}
    {\Large\textbf{des Aurores Boréales}\par}
    \vspace{1.5cm}
    {\large Université de Montpellier Paul Valéry | Master 2 MIASHS | Décembre 2025\par}
    \vspace{2cm}
    {\Large\textbf{Réalisé par :}\par}
    \vspace{0.5cm}
    {\large Adjimon Jérôme Vitoffodji\par}
    {\large Alvin Ingabire\par}
    
\end{titlepage}

\tableofcontents
\newpage

% ============================================
% CHAPITRE 1 : CONTEXTE ET OBJECTIFS
% ============================================

\chapter{Contexte et Objectifs du Projet}

\section{Contexte Académique}

Ce projet s'inscrit dans le cadre du Master 2 \textbf{Open Data et Web des Données} à l'université de Montpellier Paul Valéry, formation Master 2 MIASHS. L'objectif était de créer une application web interactive exploitant des données en temps réel provenant d'APIs publiques, en mettant l'accent sur la visualisation de données scientifiques et météorologiques.

\section{Inspiration et Choix du Sujet}

Le projet est inspiré du dashboard "Aurora Monitor" de \textbf{Sheila Géa}.

\section{Objectifs du Projet}

\begin{itemize}[leftmargin=*, itemsep=5pt]
    \item Créer un dashboard interactif de surveillance des aurores boréales en temps réel
    \item Intégrer multiple APIs publiques (NOAA, Open-Meteo, OpenWeatherMap, Sunrise-Sunset)
    \item Traduire l'application en français avec adaptation culturelle complète
    \item Ajouter des fonctionnalités pédagogiques (descriptions, explications scientifiques)
    \item Implémenter un système d'alertes par email intelligent et automatisé
    \item Créer une carte mondiale interactive des probabilités d'observation
    \item Documenter professionnellement le projet pour présentation académique
\end{itemize}

% ============================================
% CHAPITRE 2 : SESSION DU 29 NOVEMBRE 2025
% ============================================

\chapter{Session de Développement du 29 Novembre 2025}

\section{Vue d'Ensemble de la Session}

Cette session intensive de développement a duré environ 6 heures et a permis d'implémenter plusieurs améliorations majeures au dashboard AurorAlerte. Les travaux ont porté sur trois axes principaux :

\begin{enumerate}[leftmargin=*, itemsep=5pt]
    \item \textbf{Carte mondiale interactive} : Résolution de problèmes d'affichage et ajout d'un système de recherche de villes dynamique
    \item \textbf{Système d'alertes automatisé} : Implémentation d'un calcul automatique du seuil Kp selon la localisation de l'utilisateur
    \item \textbf{Amélioration des emails} : Enrichissement du contenu des alertes avec informations contextuelles
\end{enumerate}

\section{Problème 1 : Optimisation de la Carte Mondiale}

\subsection{Symptômes Observés}

La carte mondiale des aurores boréales s'affichait de manière compressée avec beaucoup d'espace blanc inutilisé. Plusieurs tentatives d'augmentation de la hauteur n'avaient aucun effet visible sur l'affichage final.

\begin{warnbox}
\textbf{Observation :} La carte apparaissait aplatie horizontalement avec une grande zone blanche en bas, rendant difficile la lecture des informations géographiques et des limites de visibilité des aurores.
\end{warnbox}

\subsection{Diagnostic}

Analyse approfondie des causes du problème :

\begin{enumerate}[leftmargin=*]
    \item \textbf{Projection inadaptée :} La projection cartographique "natural earth" coupait les bords du monde et ne convenait pas pour un focus sur l'hémisphère nord
    \item \textbf{Hauteur insuffisante :} Les valeurs de 600-700 pixels ne permettaient pas un affichage optimal de la zone géographique pertinente
    \item \textbf{Plage de latitude excessive :} La plage 40-90°N incluait des zones inutiles pour l'observation des aurores (Afrique du Nord, Amérique du Sud)
    \item \textbf{Absence de focus géographique :} L'affichage global du monde ne permettait pas de se concentrer efficacement sur l'hémisphère nord où les aurores sont observables
\end{enumerate}

\subsection{Solutions Explorées}

Trois approches progressives ont été testées pour résoudre le problème :

\begin{table}[H]
\centering
\small
\begin{tabular}{lcp{7cm}}
\toprule
\rowcolor{mainblue!80}
\textcolor{white}{\textbf{Version}} & \textcolor{white}{\textbf{Hauteur}} & \textcolor{white}{\textbf{Caractéristiques}} \\
\midrule
Version 1 & 900px & Augmentation de la hauteur, plage latitude 30-90°N \\
Version 2 & 1200px & Taille extra large, plage latitude 25-90°N \\
Version 3 (finale) & 800px & Focus hémisphère nord 40-85°N, projection Mercator, zoom automatique 1.5x \\
\bottomrule
\end{tabular}
\caption{Versions testées de la carte mondiale}
\end{table}

\subsection{Solution Finale Retenue}

La version finale utilise une combinaison optimale de paramètres :

\begin{itemize}[leftmargin=*]
    \item \textbf{Projection Mercator} : Meilleure lisibilité pour l'hémisphère nord que la projection equirectangular
    \item \textbf{Focus 40-85°N} : Exclusion des zones non pertinentes (hémisphère sud, tropiques)
    \item \textbf{Hauteur 800px} : Compromis optimal entre lisibilité et espace écran
    \item \textbf{Zoom 1.5x} : Agrandissement automatique de la zone d'intérêt
    \item \textbf{Centrage lat=60°, lon=0°} : Position optimale pour l'Europe du Nord et l'Arctique
\end{itemize}

\subsection{Résultats et Améliorations}

\begin{infobox}
\textbf{Améliorations Obtenues :}

\begin{itemize}[leftmargin=*]
    \item La carte remplit 100\% de la hauteur disponible sans espace blanc inutilisé
    \item Affichage exclusif des zones pertinentes : Europe du Nord, Amérique du Nord, Groenland, Arctique
    \item Suppression des zones non pertinentes : Afrique, Amérique du Sud, Antarctique, Océan Indien
    \item Bandes de latitude plus denses (tous les 1° au lieu de 5°) pour une meilleure précision visuelle
    \item Amélioration significative de la lisibilité des villes et des limites de visibilité des aurores
\end{itemize}
\end{infobox}

\section{Amélioration 2 : Système de Recherche de Villes Dynamique}

\subsection{Besoin Identifié}

Le dashboard affichait initialement 9 villes principales pré-configurées (Longyearbyen, Tromsø, Reykjavik, Stockholm, Oslo, Édimbourg, Londres, Paris, Berlin). Les utilisateurs souhaitaient pouvoir ajouter leurs propres localisations d'intérêt sans avoir à modifier le code source de l'application.

\subsection{Architecture Implémentée}

Mise en place d'un système à deux niveaux de villes :

\begin{enumerate}[leftmargin=*]
    \item \textbf{Villes principales (9 villes fixes)} : Toujours affichées automatiquement
    \begin{itemize}
        \item Représentation : Cercles noirs
        \item Couleur dynamique : Vert si aurores visibles, Rouge si invisibles
        \item Taille : 16 pixels si visible, 12 pixels si invisible
    \end{itemize}
    
    \item \textbf{Villes personnalisées (maximum 5)} : Ajoutées par recherche utilisateur
    \begin{itemize}
        \item Représentation : Losanges dorés
        \item Couleur dynamique : Jaune si aurores visibles, Orange si invisibles
        \item Taille : 14 pixels uniformément
    \end{itemize}
\end{enumerate}

\subsection{Fonctionnalités Clés}

\subsubsection{Géocodage Intelligent}

Le système utilise l'API de géocodage Open-Meteo pour convertir automatiquement les noms de villes entrés par l'utilisateur en coordonnées géographiques précises (latitude, longitude). Cette approche permet :

\begin{itemize}[leftmargin=*]
    \item Support de noms de villes en plusieurs langues
    \item Résolution automatique des ambiguïtés géographiques
    \item Gestion des variantes orthographiques
    \item Récupération des coordonnées exactes pour positionnement précis sur la carte
\end{itemize}

\subsubsection{Détection Intelligente des Doublons}

Système sophistiqué de détection des doublons utilisant deux méthodes complémentaires :

\begin{enumerate}[leftmargin=*]
    \item \textbf{Vérification par nom} : Comparaison insensible à la casse des noms de villes pour détecter les entrées identiques
    \item \textbf{Vérification par proximité géographique} : Détection de villes situées à moins de 0.5° de distance (environ 55 km), ce qui permet d'éviter l'affichage multiple de villes très proches (exemple : Paris et sa banlieue proche)
\end{enumerate}

Cette double vérification garantit qu'aucune ville n'apparaît deux fois sur la carte, améliorant la clarté de la visualisation.

\subsubsection{Messages de Feedback Utilisateur}

Système complet et contextuel de retour d'information à l'utilisateur :

\begin{itemize}[leftmargin=*]
    \item \textbf{Message de doublons} : "Ville(s) déjà présente(s) : Stockholm, Paris. Veuillez saisir d'autres villes."
    \item \textbf{Message de succès} : "3 ville(s) ajoutée(s) sur la carte !"
    \item \textbf{Message mixte} : Affichage combiné des deux types de messages si certaines villes sont ajoutées et d'autres rejetées
    \item \textbf{Message d'échec} : "Aucune ville ajoutée. Vérifiez les noms saisis." si aucune ville n'a pu être géocodée
\end{itemize}

\subsection{Statistiques Enrichies}

Ajout de quatre métriques calculées en temps réel sous la carte mondiale :

\begin{table}[H]
\centering
\small
\begin{tabular}{lp{8cm}}
\toprule
\rowcolor{mainblue!80}
\textcolor{white}{\textbf{Métrique}} & \textcolor{white}{\textbf{Description}} \\
\midrule
Latitude Limite & Latitude minimale à partir de laquelle les aurores sont théoriquement visibles, calculée selon l'indice Kp actuel \\
Distance à Limite & Distance en kilomètres entre la localisation choisie par l'utilisateur et la limite de visibilité, avec indication de direction (dans la zone visible ou distance vers le nord) \\
Aurores Ici & Indicateur binaire (OUI / NON) indiquant si les aurores sont actuellement visibles depuis la localisation choisie \\
Villes Visibles & Nombre et pourcentage de villes affichées sur la carte depuis lesquelles les aurores sont actuellement visibles \\
\bottomrule
\end{tabular}
\caption{Statistiques affichées sous la carte mondiale}
\end{table}

\subsection{Éléments Pédagogiques Ajoutés}

\subsubsection{Légende Interactive}

Création d'une légende visuelle composée de 5 cartes colorées avec dégradés CSS (hauteur 130 pixels) :

\begin{itemize}[leftmargin=*]
    \item Zone Verte : Région où les aurores sont actuellement visibles
    \item Ligne Dorée : Limite de visibilité correspondant à l'indice Kp actuel
    \item Zone Rouge : Région où les aurores ne sont pas visibles actuellement
    \item Villes Principales : Représentation des 9 villes pré-configurées (cercles)
    \item Villes Personnalisées : Représentation des villes ajoutées par l'utilisateur (losanges)
\end{itemize}

\subsubsection{Tableau d'Interprétation Kp}

Tableau pédagogique de 10 lignes (Kp 0 à 9) avec mise en forme conditionnelle dynamique :

\begin{itemize}[leftmargin=*]
    \item \textbf{Ligne actuelle} (Kp en cours) : Surlignée en gradient vert pour identification immédiate
    \item \textbf{Lignes passées} (Kp inférieur à l'actuel) : Fond vert clair indiquant des conditions déjà atteintes
    \item \textbf{Lignes futures} (Kp supérieur à l'actuel) : Fond rouge clair indiquant des conditions nécessitant plus d'activité géomagnétique
\end{itemize}

Colonnes du tableau : Indice Kp, Latitude limite, Régions géographiques visibles, Fréquence d'observation

\section{Amélioration 3 : Système d'Alertes Email Automatisé}

\subsection{Problématique Initiale}

Dans la version précédente du système d'alertes, l'utilisateur devait accomplir manuellement trois étapes :

\begin{enumerate}[leftmargin=*]
    \item Entrer son adresse email
    \item Choisir manuellement un seuil Kp sur une échelle de 0 à 9
    \item Deviner empiriquement quel indice Kp correspondait à sa ville
\end{enumerate}

\begin{warnbox}
\textbf{Problème d'Expérience Utilisateur Identifié :}

L'utilisateur moyen ne possède pas les connaissances scientifiques nécessaires pour déterminer quel seuil Kp choisir pour sa localisation géographique. Cette situation créait une confusion : "Est-ce que Kp 5.0 est approprié pour Stockholm ? Pour Paris ? Pour Tromsø ?"

Cette ambiguïté rendait le système d'alertes difficile à configurer correctement et peu intuitif, limitant son adoption et son efficacité.
\end{warnbox}

\subsection{Solution : Calcul Automatique du Seuil Kp}

\subsubsection{Principe Fondamental}

Développement d'un algorithme qui calcule automatiquement l'indice Kp minimum nécessaire pour observer les aurores boréales à une latitude géographique donnée. Cette approche élimine complètement le besoin pour l'utilisateur de comprendre la relation entre indice Kp et latitude d'observation.

\subsubsection{Table de Correspondance Scientifique}

Utilisation d'une table de correspondance basée sur les modèles scientifiques de géomagnétisme :

\begin{table}[H]
\centering
\small
\begin{tabular}{ccc}
\toprule
\rowcolor{mainblue!80}
\textcolor{white}{\textbf{Indice Kp}} & \textcolor{white}{\textbf{Latitude (°N)}} & \textcolor{white}{\textbf{Exemples de Villes}} \\
\midrule
0 & 66.5 & Cercle arctique \\
1 & 64.5 & Nord de l'Islande \\
2 & 62.4 & Laponie \\
3 & 60.4 & Rovaniemi (Finlande) \\
4 & 58.3 & Stockholm, Helsinki \\
5 & 56.3 & Écosse \\
6 & 54.2 & Nord de l'Angleterre, Danemark \\
7 & 52.2 & Londres, Amsterdam \\
8 & 50.1 & Bruxelles, Nord de la France \\
9 & 48.1 & Paris, Munich \\
\bottomrule
\end{tabular}
\caption{Correspondance KpLatitude minimale}
\end{table}

\subsubsection{Exemples de Calcul Automatique}

Le système calcule automatiquement le seuil optimal selon la ville choisie :

\begin{itemize}[leftmargin=*]
    \item \textbf{Stockholm} (59.33°N) → Kp minimum calculé : 4
    \item \textbf{Tromsø} (69.65°N) → Kp minimum calculé : 1
    \item \textbf{Paris} (48.85°N) → Kp minimum calculé : 9
    \item \textbf{Londres} (51.51°N) → Kp minimum calculé : 8
\end{itemize}

\subsection{Nouvelle Interface Utilisateur Simplifiée}

\subsubsection{Workflow Optimisé}

Le processus de configuration des alertes est désormais réduit à 4 étapes simples :

\begin{enumerate}[leftmargin=*]
    \item L'utilisateur sélectionne sa ville dans l'interface (exemple : Stockholm)
    \item Le système calcule automatiquement et silencieusement : Kp minimum = 4
    \item L'utilisateur entre son adresse email et valide
    \item Le système affiche un message contextuel : "Bonne localisation ! Aurores régulières (Kp >= 4)"
\end{enumerate}

\subsubsection{Messages Contextuels Intelligents}

Le système génère des messages personnalisés en fonction du Kp minimum calculé, offrant une évaluation qualitative de la localisation pour l'observation des aurores :

\begin{table}[H]
\centering
\small
\begin{tabular}{ccp{6.5cm}}
\toprule
\rowcolor{mainblue!80}
\textcolor{white}{\textbf{Kp Min}} & \textcolor{white}{\textbf{Type Message}} & \textcolor{white}{\textbf{Message Affiché}} \\
\midrule
0-2 & Succès (vert) & "Excellente localisation ! Aurores fréquentes. Vous en verrez souvent !" \\
3-5 & Info (bleu) & "Bonne localisation ! Les aurores sont régulièrement visibles ici." \\
6-7 & Avertissement (jaune) & "Aurores rares ici. Tempêtes géomagnétiques nécessaires. Profitez de cette occasion !" \\
8-9 & Erreur (rouge) & "Aurores très rares. Événements géomagnétiques extrêmes requis. Conseil : Voyagez plus au nord !" \\
\bottomrule
\end{tabular}
\caption{Messages personnalisés selon le Kp minimum}
\end{table}

\subsubsection{Paramètres Avancés pour Utilisateurs Expérimentés}

Un panneau extensible "Paramètres Avancés" permet aux utilisateurs expérimentés de :

\begin{itemize}[leftmargin=*]
    \item Activer la personnalisation manuelle du seuil Kp (désactivée par défaut)
    \item Modifier l'intervalle minimum entre deux alertes (0.5 à 6 heures, défaut 1 heure)
    \item Visualiser le calcul automatique pour comprendre le raisonnement du système
\end{itemize}

Par défaut, ce panneau reste caché et le mode automatique intelligent est utilisé, simplifiant l'expérience pour 95\% des utilisateurs.

\subsection{Avantages de l'Approche Automatisée}

\begin{infobox}
\textbf{Améliorations Significatives de l'Expérience Utilisateur :}

\begin{itemize}[leftmargin=*]
    \item Simplicité accrue : Élimination de la configuration manuelle du seuil Kp technique
    \item Cohérence logique : Alignement avec le reste du dashboard où la localisation est centrale
    \item Intelligence contextuelle : Calcul automatique précis basé sur des données scientifiques
    \item Fiabilité maximale : Impossibilité de mal configurer le système
    \item Valeur pédagogique : L'utilisateur comprend pourquoi ce seuil spécifique est approprié
    \item Professionnalisme : L'application s'adapte à l'utilisateur, pas l'inverse
\end{itemize}
\end{infobox}

\subsection{Correction d'un Bug Architectural Critique}

\subsubsection{Erreur Détectée}

Une erreur système apparaissait lors du chargement de l'application : \texttt{NameError: name 'kp\_zones' is not defined}

\subsubsection{Analyse de la Cause}

Investigation révélant que la variable globale \texttt{kp\_zones} était définie à l'intérieur d'un bloc conditionnel spécifique (onglet "Carte mondiale") mais utilisée dans une section exécutée avant (système d'alertes).

\textbf{Principe d'architecture violé :} En Python/Streamlit, toute variable utilisée dans plusieurs endroits du code doit être définie GLOBALEMENT au début de l'application, avant toute logique conditionnelle ou de navigation par onglets.

\subsubsection{Solution Implémentée}

Déplacement de la définition de \texttt{kp\_zones} vers le début du fichier principal, immédiatement après la configuration de la page Streamlit, transformant ainsi cette table en constante globale accessible depuis n'importe quel point de l'application.

Cette correction a également amélioré la maintenabilité du code en clarifiant que cette table est une donnée scientifique fondamentale du projet, pas une configuration spécifique à un onglet.

\section{Amélioration 4 : Enrichissement des Emails d'Alerte}

\subsection{Modification de l'Architecture Email}

Le système d'envoi d'email a été enrichi pour inclure automatiquement le contexte de localisation de l'utilisateur, transformant une alerte générique en notification personnalisée et pédagogique.

\subsection{Nouveau Contenu Email Personnalisé}

\subsubsection{Section Informative Contextuelle}

Ajout d'une section dédiée dans chaque email expliquant le contexte géographique :

\begin{itemize}[leftmargin=*]
    \item \textbf{Pour localisations excellentes} (Kp min 0-2) : "Excellente nouvelle ! Votre localisation est idéale pour observer les aurores. Vous en verrez souvent !"
    \item \textbf{Pour localisations bonnes} (Kp min 3-5) : "Bonne localisation ! Les aurores sont régulièrement visibles ici."
    \item \textbf{Pour localisations rares} (Kp min 6-7) : "Les aurores sont rares à cette latitude. Profitez de cette occasion exceptionnelle !"
    \item \textbf{Pour localisations très rares} (Kp min 8-9) : "Événement exceptionnel ! Les aurores sont très rares ici. Ne manquez pas ce spectacle unique !"
\end{itemize}

\subsubsection{Nouvelles Métriques Affichées}

L'email HTML enrichi présente désormais 5 métriques clés au lieu de 4 :

\begin{enumerate}[leftmargin=*]
    \item \textbf{Indice Kp Actuel} : Valeur en temps réel de l'activité géomagnétique (sur 9)
    \item \textbf{Kp Minimum Requis} : Seuil calculé spécifiquement pour cette localisation (NOUVEAU)
    \item \textbf{Score de Probabilité} : Score composite prenant en compte Kp, météo et obscurité (sur 1.0)
    \item \textbf{Ciel Dégagé} : Pourcentage de ciel sans nuages
    \item \textbf{Obscurité} : Indicateur nuit/jour avec icône appropriée
\end{enumerate}

\subsection{Structure de l'Email Amélioré}

\subsubsection{Format HTML Professionnel}

L'email utilise un template HTML responsive avec :

\begin{itemize}[leftmargin=*]
    \item \textbf{En-tête dynamique} : Couleur de fond adaptée au score (vert pour excellent, jaune pour bon, rouge pour moyen)
    \item \textbf{Section d'information contextuelle} : Fond jaune clair avec bordure pour attirer l'attention sur le message personnalisé
    \item \textbf{Métriques visuelles} : Présentation en grille des 5 indicateurs clés avec typographie hiérarchisée
    \item \textbf{Conseils d'observation} : Liste détaillée (meilleure période, lieu idéal, direction, patience, conseils photo)
    \item \textbf{Bouton d'action} : Lien cliquable vers le dashboard complet
    \item \textbf{Footer informatif} : Timestamp UTC précis et informations sur l'origine de l'alerte
\end{itemize}

\subsubsection{Version Texte Alternative}

Pour garantir la compatibilité avec tous les clients email, chaque email inclut également une version texte brut contenant toutes les informations essentielles dans un format lisible sans HTML.

\section{Tests et Validation du Système d'Alertes}

\subsection{Création d'Outils de Test Dédiés}

Pour faciliter les tests du système d'alertes email sans nécessiter l'activation complète du dashboard, un script Python dédié a été développé : \texttt{test\_email.py}

\subsubsection{Fonctionnalités du Script de Test}

\begin{itemize}[leftmargin=*]
    \item Chargement automatique de la configuration SMTP depuis le fichier de secrets
    \item Interface interactive en ligne de commande pour entrer l'email de test
    \item Validation automatique du format de l'adresse email
    \item Envoi d'un email avec des données de test prédéfinies (Kp 6.5, Stockholm, score 0.82)
    \item Affichage de messages de debug détaillés pour faciliter le diagnostic
    \item Fourniture automatique de solutions en cas d'erreur d'envoi
\end{itemize}

\subsection{Méthodes de Test Documentées}

Trois approches complémentaires de test ont été documentées :

\begin{table}[H]
\centering
\small
\begin{tabular}{lp{8.5cm}}
\toprule
\rowcolor{mainblue!80}
\textcolor{white}{\textbf{Méthode}} & \textcolor{white}{\textbf{Description et Usage}} \\
\midrule
Script test\_email.py & Script Python standalone avec interface interactive, idéal pour tests rapides et debugging \\
Test depuis Dashboard & Modification temporaire du seuil Kp à 0.0 pour forcer l'envoi immédiat d'une alerte de test \\
Test interactif Python & Console Python avec import manuel, utile pour tests unitaires de la fonction d'envoi \\
\bottomrule
\end{tabular}
\caption{Méthodes de test disponibles}
\end{table}

\subsection{Guide de Configuration Gmail}

\subsubsection{Processus de Configuration Documenté}

Un guide détaillé a été créé pour configurer Gmail avec mot de passe d'application :

\begin{enumerate}[leftmargin=*]
    \item Accéder aux paramètres de sécurité du compte Google
    \item Activer l'authentification à deux facteurs (prérequis obligatoire)
    \item Générer un mot de passe d'application spécifique pour "AurorAlerte"
    \item Copier le mot de passe généré (16 caractères) dans le fichier de configuration
    \item Tester la connexion avec le script de test
\end{enumerate}

\begin{warnbox}
\textbf{Note Importante de Sécurité :}

Il est impératif d'utiliser un \textbf{mot de passe d'application} Gmail et non le mot de passe principal du compte. Cette pratique garantit :
\begin{itemize}[leftmargin=*]
    \item Meilleure sécurité (le mot de passe peut être révoqué sans changer le mot de passe principal)
    \item Traçabilité des accès
    \item Conformité avec les recommandations de sécurité de Google
\end{itemize}
\end{warnbox}

\subsection{Documentation des Problèmes Courants}

Création d'un guide de résolution des erreurs fréquemment rencontrées :

\begin{table}[H]
\centering
\small
\begin{tabular}{lp{7cm}}
\toprule
\rowcolor{mainblue!80}
\textcolor{white}{\textbf{Erreur Rencontrée}} & \textcolor{white}{\textbf{Solution Documentée}} \\
\midrule
Authentification échouée & Vérifier l'utilisation d'un mot de passe d'application et l'activation de l'authentification à deux facteurs \\
Connection refused / Timeout & Désactiver temporairement l'antivirus, autoriser Python dans le pare-feu, vérifier que le port 587 n'est pas bloqué \\
Invalid sender & Confirmer que l'adresse email expéditeur correspond exactement au compte Gmail utilisé \\
Email non reçu & Attendre 5-10 minutes, vérifier minutieusement le dossier SPAM/Courrier indésirable \\
\bottomrule
\end{tabular}
\caption{Guide de résolution des problèmes}
\end{table}

\section{Livrables de la Session}

\subsection{Fichiers de Développement Créés}

Au total, 11 fichiers ont été créés pendant cette session :

\begin{table}[H]
\centering
\small
\begin{tabular}{lp{7cm}}
\toprule
\rowcolor{mainblue!80}
\textcolor{white}{\textbf{Fichier}} & \textcolor{white}{\textbf{Contenu et Utilité}} \\
\midrule
carte\_mercator.py & Première version corrigée (900px) \\
carte\_mercator\_1200px.py & Version expérimentale extra large \\
carte\_hemisphere\_nord.py & Version finale optimisée recommandée \\
carte\_avec\_recherche\_villes.py & Carte complète avec système de recherche \\
alertes\_avec\_validation.py & Système d'alertes avec validation email \\
alertes\_automatiques.py & Version finale automatisée complète \\
alerts\_ameliore.py & Fichier model/alerts.py enrichi \\
kp\_zones\_definition.py & Définition globale de la table Kp-Latitude \\
test\_email.py & Script de test du système d'alertes \\
\bottomrule
\end{tabular}
\caption{Fichiers Python créés}
\end{table}

\subsection{Documentation Technique Produite}

\begin{table}[H]
\centering
\small
\begin{tabular}{lp{7cm}}
\toprule
\rowcolor{mainblue!80}
\textcolor{white}{\textbf{Document}} & \textcolor{white}{\textbf{Contenu}} \\
\midrule
INTEGRATION\_ALERTES\_AUTO.md & Guide complet d'intégration du système automatisé (40+ pages) \\
GUIDE\_TEST\_EMAIL.md & Documentation complète des tests et résolution de problèmes (25+ pages) \\
\bottomrule
\end{tabular}
\caption{Documentation créée}
\end{table}

\section{Statistiques de la Session}

\begin{table}[H]
\centering
\begin{tabular}{lr}
\toprule
\rowcolor{mainblue!80}
\textcolor{white}{\textbf{Métrique}} & \textcolor{white}{\textbf{Valeur}} \\
\midrule
Durée totale de la session & environ 6 heures \\
Problèmes majeurs résolus & 4 \\
Nouvelles fonctions créées & 3 \\
Lignes de code ajoutées/modifiées & environ 500 \\
Fichiers créés (code + docs) & 11 \\
Captures d'écran analysées & 3 \\
Versions de carte testées & 3 \\
Types de messages utilisateur & 8 \\
Pages de documentation & 65+ \\
\bottomrule
\end{tabular}
\caption{Statistiques détaillées de la session}
\end{table}

\chapter{Technologies et APIs}
\section{Stack Technique}

\begin{table}[H]
\centering
\begin{tabular}{lcp{7cm}}
\toprule
\rowcolor{mainblue!80}
\textcolor{white}{\textbf{Technologie}} & \textcolor{white}{\textbf{Version}} & \textcolor{white}{\textbf{Utilisation}} \\
\midrule
Python & 3.10+ & Langage principal du backend \\
Streamlit & 1.28+ & Framework web interactif \\
Plotly & 5.17+ & Visualisations graphiques interactives \\
Pandas & 2.1+ & Manipulation et analyse de données \\
Requests & 2.31+ & Appels APIs REST \\
smtplib & built-in & Envoi d'emails via SMTP \\
PIL/Pillow & 10.0+ & Traitement d'images \\
\bottomrule
\end{tabular}
\caption{Technologies principales utilisées}
\end{table}

\section{APIs Intégrées}

\begin{itemize}[leftmargin=*]
    \item NOAA Space Weather Prediction Center (Kp, prévisions, animations OVATION)
    \item Open-Meteo (météo, géocodage)
    \item OpenWeatherMap (météo actuelle, icônes)
    \item Sunrise-Sunset (obscurité)
\end{itemize}

\chapter{Fonctionnalités Implémentées}

\begin{itemize}[leftmargin=*]
    \item 3 jauges interactives temps réel
    \item Historique Kp 4 heures avec export CSV
    \item Carte mondiale Mercator focalisée hémisphère nord (40-85°N)
    \item Recherche de 5 villes personnalisées + détection automatique des doublons
    \item Calcul automatique du seuil Kp selon la localisation
    \item Messages contextuels personnalisés (4 niveaux)
    \item Emails HTML enrichis avec 5 métriques
    \item Documentation technique complète (65+ pages)
\end{itemize}

\chapter{Tests et Validation}

\section{Tests Fonctionnels}

\begin{table}[H]
\centering
\small
\begin{tabular}{lc}
\toprule
\rowcolor{mainblue!80}
\textcolor{white}{\textbf{Test Effectué}} & \textcolor{white}{\textbf{Résultat}} \\
\midrule
Traduction complète français & Passé \\
4 APIs fonctionnelles & Passé \\
Géocodage français (Suède, Norvège) & Passé \\
Alertes email (test forcé) & Passé \\
Carte zones Kp 3, 5, 7, 9 & Passé \\
Recherche villes + doublons & Passé \\
Calcul Kp auto (3 villes testées) & Passé \\
Export CSV & Passé \\
Responsive 2 résolutions & Passé \\
Gestion erreurs & Passé \\
\bottomrule
\end{tabular}
\caption{Résultats tests fonctionnels}
\end{table}

\chapter{Améliorations Futures}

\begin{itemize}[leftmargin=*, itemsep=5pt]
    \item Historique observations utilisateur (BDD), Alertes SMS Twilio, Favoris localisations multiples
    \item Mode hors ligne (cache), Webcams additionnelles, Support multilingue (EN, SV, NO), Timeline Kp 24h
    \item Prédictions ML, Application mobile, Partage social
\end{itemize}

\chapter{Conclusion}

\section{Bilan des Objectifs}

Le projet AurorAlerte a dépassé ses objectifs initiaux en créant un dashboard professionnel combinant :

\begin{itemize}[leftmargin=*, itemsep=5pt]
    \item 4 APIs temps réel parfaitement intégrées
    \item Traduction française complète et localisée
    \item 15+ emplacements pédagogiques détaillés
    \item Système alertes intelligent automatisé
    \item Carte mondiale avec recherche dynamique
    \item Export données multiples formats
    \item 9 problèmes techniques majeurs documentés et résolus
    \item Guide test complet (65+ pages documentation)
\end{itemize}

\section{Compétences Acquises}

\begin{infobox}
\textbf{Compétences Techniques :} Python/Streamlit avancé, APIs REST multiples, Plotly visualisation géographique, SMTP/Gmail sécurisé, Pandas manipulation données, Gestion états Streamlit, Projections cartographiques, Géocodage multilingue

\vspace{0.3cm}
\textbf{Méthodologie Professionnelle :} Debugging systématique, Documentation technique exhaustive, Gestion version Git, Architecture logicielle modulaire, Tests fonctionnels, Scripts test autonomes, Résolution problèmes utilisateur

\vspace{0.3cm}
\textbf{Soft Skills Développés :} Adaptation code existant, Traduction/localisation culturelle, Pédagogie scientifique, Gestion projet académique, UX design centré utilisateur, Feedback contextuel, Communication technique
\end{infobox}

\section{Réflexion Finale}

Ce projet s'est étendu sur plusieurs semaines avec une session intensive finale le 29 novembre 2025. L'approche choisie d'adapter un code existant plutôt que repartir de zéro s'est révélée extrêmement formatrice, reflétant fidèlement les pratiques professionnelles réelles.

La session du 29 novembre a particulièrement démontré l'importance d'une méthodologie rigoureuse : analyse détaillée des problèmes avec documentation visuelle, exploration de solutions multiples avant décision, tests systématiques à chaque étape, et documentation exhaustive pour reproductibilité.

L'aspect le plus gratifiant du projet a été sa dimension pédagogique : transformer des concepts scientifiques complexes (géomagnétisme, météorologie, projections géographiques) en expérience utilisateur accessible et engageante.

La refonte du système d'alertes illustre parfaitement le principe central du design centré utilisateur : le système doit s'adapter intelligemment à l'utilisateur, jamais l'inverse.

\begin{infobox}
\textbf{AurorAlerte} représente désormais un outil opérationnel et pédagogique complet, prêt au déploiement pour passionnés d'aurores boréales francophones.

Le code source documenté, le README détaillé et ce journal de bord exhaustif permettent à d'autres développeurs de contribuer et d'étendre le projet.

L'application démontre qu'il est possible de créer des outils scientifiques qui soient simultanément accessibles, esthétiques, rigoureux et pédagogiques en combinant APIs publiques, frameworks modernes et attention méticuleuse au design d'expérience utilisateur.
\end{infobox}

% ============================================
% ANNEXES
% ============================================

\appendix

\chapter{Statistiques Finales du Projet}

\begin{table}[H]
\centering
\begin{tabular}{lr}
\toprule
\rowcolor{mainblue!80}
\textcolor{white}{\textbf{Métrique}} & \textcolor{white}{\textbf{Valeur}} \\
\midrule
Lignes de code Python & environ 2000 \\
Lignes de documentation & environ 2500 \\
Fonctions développées & 30+ \\
APIs intégrées & 4 \\
Langues supportées & 2 (FR + EN pays) \\
Onglets interface & 7 \\
Visualisations Plotly & 20+ \\
Temps développement total & environ 50 heures \\
Commits Git & 40+ \\
Problèmes techniques résolus & 9 \\
Tests fonctionnels documentés & 10 \\
Pages documentation technique & 65+ \\
\bottomrule
\end{tabular}
\caption{Statistiques complètes (mise à jour 29/11/2025)}
\end{table}

\chapter{Ressources et Liens}

\begin{itemize}[leftmargin=*]
    \item Site web : \url{https://web-production-ff2d6.up.railway.app/#aura-hunter}
    \item Version Github : \url{https://github.com/JeromeVitoff/Open_Data-Web_donnees}
    \item NOAA SWPC : \url{https://www.swpc.noaa.gov/}
    \item Open-Meteo : \url{https://open-meteo.com/}
    \item OpenWeatherMap : \url{https://openweathermap.org/}
    \item Sunrise-Sunset : \url{https://sunrise-sunset.org/api}
    \item Streamlit : \url{https://docs.streamlit.io/}
\end{itemize}

\vfill

\begin{center}
\textbf{Document réalisé par :}\\
\vspace{0.3cm}
Adjimon Jérôme Vitoffodji et Alvin Ingabire\\
\vspace{0.8cm}
Master 2 MIASHS, Open Data et Web des Données\\
\end{center}

\end{document}